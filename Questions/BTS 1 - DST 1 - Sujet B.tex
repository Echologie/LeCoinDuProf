\documentclass[oneside,twocolumn,landscape]{book}
\usepackage[T1]{fontenc}
\usepackage[utf8]{inputenc}
\usepackage{geometry}
\geometry{verbose,tmargin=1cm,bmargin=1cm,lmargin=2cm,rmargin=2cm}
\setcounter{secnumdepth}{3}
\setcounter{tocdepth}{3}
\usepackage{amsmath}
\usepackage{amssymb}
\usepackage{amsfonts}
\usepackage{bbold}
\pagestyle{empty}

\begin{document}

\section*{Contrôle de Mathématiques}

\let\MauvaiseReponse\null
\let\BonneReponse\null

{\bf Merci de répondre sur la grille fournie et de ne pas oublier de traiter les deux questions au verso en évitant de sortir des cadres. Vous prendrez soin de rendre le sujet avec cette grille.}
\vspace{2em}







\begin{enumerate}




\item L'ensemble des solutions de l'inéquation $-3 x+2<5$ est :

\begin{enumerate}

\item\MauvaiseReponse $\mathbb{R}$

\item\BonneReponse $]-1,+\infty[$

\item\MauvaiseReponse $]-\infty, 1]$

\end{enumerate}

\item L'ensemble des solutions de l'inéquation $-3x+8 \geqslant 5$

\begin{enumerate}

\item\MauvaiseReponse $\mathbb{R}$

\item\MauvaiseReponse $[1,+\infty[$

\item\BonneReponse $]-\infty,1]$

\end{enumerate}





\item Le nombre $\frac{1}{6}$

\begin{enumerate}

\item\MauvaiseReponse est solution de l'équation $6x+1=0$

\item\MauvaiseReponse est solution de l'équation $x+6=0$

\item\BonneReponse est solution de l'inéquation $3x+3>0$

\end{enumerate}



\item Le nombre $\sqrt{6}$

\begin{enumerate}

\item\MauvaiseReponse est solution de l'équation $x^{2}+6=0$

\item\BonneReponse est solution de l'équation $x^{3}-6x=0$

\item\MauvaiseReponse est solution de l'inéquation $-6x+2>0$

\end{enumerate}

\newpage

\item Le nombre $\frac{1}{8}$

\begin{enumerate}

\item\BonneReponse est solution de l'équation $8x-1=0$
\item\MauvaiseReponse est solution de l'équation $7x+1=0$


\item\MauvaiseReponse est solution de l'inéquation $8x+3<0$

\end{enumerate}



\item Le couple solution du système $\left\{\begin{array}{c}5x+2y=-7 \\ x-8y=-35\end{array}\right.$ est

\begin{enumerate}

\item\MauvaiseReponse $(6;4)$

\item\BonneReponse $(-3 ;4)$
\item\MauvaiseReponse $(2 ; -3)$


\end{enumerate}






\item $u$ est la suite définie pour tout entier $n \geqslant 1$ par $u_{n}=\frac{-2n^{2}-4}{n^{2}}$.

\begin{enumerate}

\item\MauvaiseReponse $u_{3}=3$


\item\MauvaiseReponse $u_{3} =\frac{32}{9}$
\item\BonneReponse $u_{3}=-\frac{22}{9}$

\end{enumerate}



\item Soit $\left(u_{n}\right)$ la suite définie par: $u_{0}=2$ et, pour tout entier naturel $n$, $u_{n+1}=3 u_{n}-3$, alors :

\begin{enumerate}

\item\MauvaiseReponse $u_{3}=2$

\item\BonneReponse $u_{3}=15$
\item\MauvaiseReponse $u_{3}=6$


\end{enumerate}



\newpage

\item $v$ est la suite définie par $v_{0}=-2$ et la relation de récurrence\\ $v_{n+1}=4n-v_{n}$, alors :

\begin{enumerate}


\item\MauvaiseReponse $v_{2}=-5$
\item\BonneReponse $v_{2}=2$

\item\MauvaiseReponse $v_{2}=5$

\end{enumerate}



\item On considère la suite arithmétique $\left(u_{n}\right)$ de premier terme $u_{0}=-3$\\ et de raison $r=-\frac{1}{2}$.

\begin{enumerate}

\item\MauvaiseReponse $u_{3}=-3$

\item\BonneReponse $u_{3}=-\frac{9}{2}$
\item\MauvaiseReponse $u_{3}=-4$


\end{enumerate}



\item Soit $\left(u_{n}\right)$ une suite géométrique de raison $q=\frac{1}{4}$. Sachant que $u_{3}=\frac{5}{8}$,\\ le premier terme $\mathrm{u}_ {0}$ est :

\begin{enumerate}


\item\MauvaiseReponse $\frac{5}{512}$
\item\BonneReponse $40$

\item\MauvaiseReponse $\frac{128}{125}$

\end{enumerate}



\item Soit la suite $\left(u_{n}\right)$ définie par $u_{n+1}=\frac{1}{6}u_n$. Cette suite est :

\begin{enumerate}

\item\BonneReponse décroissante
\item\MauvaiseReponse croissante


\item\MauvaiseReponse constante

\end{enumerate}

\newpage


\item On considère l'équation $2x^{2}-5 x-9=0$ alors le discriminant $\Delta$ est égal à :

\begin{enumerate}


\item\MauvaiseReponse $-47$
\item\BonneReponse $97$

\item\MauvaiseReponse $11$

\end{enumerate}


\item Le nombre de solutions de l'équation $3x^{2}-48 x+192=0$ est

\begin{enumerate}

\item\MauvaiseReponse $0$


\item\MauvaiseReponse $2$
\item\BonneReponse $1$

\end{enumerate}



\item L'ensemble des solutions de l'équation $x^{2}+3 x-28=0$ est :

\begin{enumerate}

\item\BonneReponse $\{4 ;-7\}$
\item\MauvaiseReponse $\emptyset$


\item\MauvaiseReponse $\{4 ; 7\}$

\end{enumerate}






\end{enumerate}

\end{document}
